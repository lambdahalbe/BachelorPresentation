\documentclass[11pt]{beamer}
\usepackage[utf8]{inputenc}
\usepackage[ngerman]{babel}
\usepackage{amsmath}
\usepackage{amsfonts}
\usepackage{amssymb}
\usepackage{graphicx}
\usetheme{Copenhagen}
\usepackage{paralist}


\author{Fabian Glatzel}
\title[Bachelor Vortrag]{Berechnung von Tight-Binding Parametern mit Dichtefunktionaltheorie am Beispiel von \emph{trans}-Polyacetylen}
\subtitle{Bachelor-Vortrag}
%\logo{}
\institute{Physikalisches Institut}
\date{\today}
%\subject{}
\setbeamercovered{transparent}
\setbeamertemplate{navigation symbols}{}

\begin{document}
\begin{frame}[plain]
	\maketitle
\end{frame}

\begin{frame}
\tableofcontents
\end{frame}

\section{Einleitung}
\begin{frame}
\frametitle{Motivation}
\begin{itemize}
\item Große Anwendungsbreite von organischen Halbleitern
\item Polyacetylen als einfaches Testsystem
\item 1950-er \textsc{Longuet-Higgins}
\end{itemize}
\end{frame}

\section{Ergebnisse}
\begin{frame}
	\frametitle{}
\end{frame}

\section{Zusammenfassung}
\begin{frame}
	\frametitle{title}
\end{frame}
\end{document}